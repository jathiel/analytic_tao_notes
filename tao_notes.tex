\documentclass[11pt]{article}

\usepackage{amsmath}
\usepackage{amsthm}
\usepackage{amsfonts}

\pdfpagewidth 8.5in
\pdfpageheight 11in
\topmargin -1in
\headheight 0in
\headsep 0in
\textheight 8.5in
\textwidth 6.5in
\oddsidemargin 0in
\evensidemargin 0in
\headheight 77pt
\headsep 0in
\footskip .75in

\allowdisplaybreaks

\newtheorem{innercustomthm}{Exercise}
\newenvironment{ex}[1]
  {\renewcommand\theinnercustomthm{#1}\innercustomthm}
  {\endinnercustomthm}

\begin{document}

\title{Comments, Notes, and Problem Solutions for Terry Tao's Analytic Number Theory Notes}
\date{}
\maketitle

\section{Summing monotone functions}

\begin{ex}{4}\label{four}
For non-negative numbers $k,l\geq0$, show that
\begin{align}
\sum_{n\leq x}\log^k{n}\log^l{\frac{x}{n}} &= xP_{k,l}(\log{x})+O_{k,l}(\log^{k+l}{(2+x)})\label{four_result}
\end{align}
for all $x\geq 0$ and some polynomal $P_{k,l}(t)$ with leading term $l!t^k$.
\end{ex}

\begin{proof}
We prove Exercise~\ref{four} by induction on $l$. Furthermore, we will show that
\begin{align}
P_{k,l}(t) &= \sum_{i=0}^k(-1)^i(l+i)!\binom{k}{i}t^{k-i}.\label{pkl_poly}
\end{align}

We begin with the case where $l=0$. By Lemma 2, we have
\begin{align}
\sum_{n\leq x}\log^k{n} &= \int_1^x\log^k{t}\;dt + O_k(\log^k(2+x))\label{log_base_case}
\end{align}
for $x\geq 0$. Using integration by parts, for any $k\geq0$, we have
\begin{align}
\int_1^x\log^k{t}\;dt &= t\log^k{t}\Big\vert_1^x-k\int_1^x\log^{k-1}{t}\;dt.\label{log_by_parts}
\end{align}
It now follows by induction on~\eqref{log_by_parts} that
\begin{align}
\int_1^x\log^k{t}\;dt &= x\log^k{x}-kx\log^{k-1}{x}+k(k-1)x\log^{k-1}{x}+\cdots+(-1)^kk!(x-1)\notag\\
&= x\big(\log^k{x}-k\log^{k-1}{x}+\cdots+(-1)^kk!\big)+(-1)^{k+1}k!\notag\\
&= x\sum_{i=0}^{k}(-1)^i\frac{k!}{(k-i)!}\log^{k-j}{x}+(-1)^{k+1}k!\notag\\
&= xP_{k,0}(\log{x})+O_k(1).\label{log_poly}
\end{align}
Combining~\eqref{log_base_case} with~\eqref{log_poly} shows that~\eqref{four_result} holds in this case.

Now we assume that~\eqref{four_result} holds for some value $l-1\geq 0$. Then, by the induction hypothesis, we have
\begin{align}
\sum_{n\leq x}\log^k{n}\log^l{\frac{x}{n}} &= \sum_{n\leq x}(\log^k{n})(\log{x}-\log{n})^l\notag\\
&= \sum_{n\leq x}(\log^k{n})(\log{x}-\log{n})^{l-1}(\log{x}-\log{n})\notag\\
&= x(\log{x}\cdot P_{k,l-1}(\log{x})-P_{k+1,l-1}(\log{x}))+O_{k,l}(\log^{k+l}(2+x))\label{log_diff}
\end{align}
for $x\geq 0$. From~\eqref{log_diff}, we see that in order to complete the proof, we must show that
\begin{align*}
t\cdot P_{k,l-1}(t)-P_{k+1,l-1}(t) &= P_{k,l}(t).
\end{align*}
From~\eqref{pkl_poly}, we have
\begin{align*}
t\cdot P_{k,l-1}(t)-P_{k+1,l-1}(t) &= \sum_{i=0}^k(-1)^i(l-1+i)!\binom{k}{i}t^{k+1-i}-\sum_{i=0}^{k+1}(-1)^i(l-1+i)!\binom{k+1}{i}t^{k+1-i}\\
&= \sum_{i=0}^k(-1)^i(l-1+i)!\bigg(\binom{k}{i}-\binom{k+1}{i}\bigg)t^{k+1-i}+(-1)^k(l+k)!\\
&= \sum_{i=1}^k(-1)^{i+1}(l-1+i)!\binom{k}{i-1}t^{k+1-i}+(-1)^k(l+k)!\\
&= \sum_{i=0}^{k-1}(-1)^i(l+i)!\binom{k}{i}t^{k-i}+(-1)^k(l+k)!\\
&= P_{k,l}(t).
\end{align*}
Therefore~\eqref{four_result} and~\eqref{pkl_poly} hold in the general case.
\end{proof}

\begin{ex}{6}\label{six}
Let $f:\mathbb{N}\to\mathbb{C}$, $F:\mathbb{R}^+\to\mathbb{C}$ and $g:\mathbb{R}^+\to\mathbb{R}^+$ be functions such that $g(x)\to 0$ as $x\to\infty$. Then the following are equivalent:
\begin{enumerate}
\item[(i)]One has
\begin{align*}
\sum_{y\leq n<x} f(n) = F(x)-F(y)+O(g(x)+g(y))
\end{align*}
for all $1\leq y<x$.
\item[(ii)] There exists a constant $c\in\mathbb{C}$ such that
\begin{align*}
\sum_{n<x}f(n) = c+F(x)+O(g(x))
\end{align*}
for all $x\geq 1$. In particular, $c=-F(1)+O(g(1))$.
\end{enumerate}
The quantity $c$ in (ii) is unique; it is also real-valued if $f$ and $F$ are real-valued.

If in addition $F(x)\to 0$ as $x\to 0$, then when (ii) holds, $\sum_{n=1}^\infty f(n)$ converges conditionally to $c$.
\end{ex}

\begin{proof}
$(2)\Rightarrow(1):$ This direction is trivial.

$(1)\Rightarrow(2):$ Let
\begin{align}
E(x):= \sum_{n<x} f(n) - F(x).\label{e_def}
\end{align}
Our goal is to show that~\eqref{e_def} converges to some constant as $x\to\infty$. Suppose that this is not the case. Then there is some $\epsilon >0$ and two strictly increasing sequences $\{x_n\}_{n=1}^\infty$, $\{y_n\}_{n=1}^\infty$ with $x_n > y_n\geq 1$ for all $n$ such that $|E(x_n)-E(y_n)| \geq \epsilon$. However, by assumption, we have that $E(x_n)-E(y_n)=O(g(x_n)+g(y_n))$, which converges to 0 as $n\to\infty$. This contradiction gives the desired result.
\end{proof}

\begin{ex}{7}\label{seven}
Let $f$ be an arithmetic function. If the natural density $\lim_{x\to\infty}\frac{1}{x}\sum_{n\leq x}f(n)$ of $f$ exists and is equal to some complex number $\alpha$, show that the logarithmic density $\lim_{x\to\infty}\frac{1}{\log{x}}\sum_{n\leq x}\frac{f(n)}{n}$ also exists and is equal to $\alpha$.
\end{ex}

\begin{proof}
We begin by establishing the following identity:
\begin{align}
\sum_{n\leq x}\frac{f(n)}{n} &= \frac{1}{x}\sum_{n\leq x}f(n)+\int_1^x\frac{1}{t}\sum_{n\leq t}f(n)\frac{dt}{t}.\label{log_identity}
\end{align}
By splitting up the integral on the right-hand side of~\eqref{log_identity}, we get that
\begin{align}
\int_1^x\frac{1}{t^2}\sum_{n\leq t}f(n)\;dt &= \sum_{m=1}^{[x]-1}\int_m^{m+1}\frac{1}{t^2}\sum_{n\leq t}f(n)\;dt+\int_{[x]}^{x}\frac{1}{t^2}\sum_{n\leq t}f(n)\;dt\notag\\
&= \sum_{m=1}^{[x]-1}\int_m^{m+1}\frac{1}{t^2}\sum_{n\leq m}f(n)\;dt+\int_{[x]}^x\frac{1}{t^2}\sum_{n\leq x}f(n)\;dt.\label{const_sum}
\end{align}
In~\eqref{const_sum}, we replace the $t$ in the sum inside the integral with $m$ because the sum is constant on the interval $[m,m+1)$. So,
\begin{align}
\int_1^x\frac{1}{t^2}\sum_{n\leq t}f(n)\;dt &= \sum_{m=1}^{[x]-1}\sum_{n\leq m}f(n)\int_m^{m+1}\frac{1}{t^2}\;dt+\sum_{n\leq x}f(n)\int_{[x]}^x\frac{1}{t^2}\;dt\notag\\
&= \sum_{m=1}^{[x]-1}\sum_{n\leq m}f(n)\bigg(\frac{1}{m}-\frac{1}{m+1}\bigg)+\sum_{n\leq x}f(n)\bigg(\frac{1}{[x]}-\frac{1}{x}\bigg)\notag\\
&= \sum_{n\leq [x]-1}f(n)\sum_{m=n}^{[x]-1}\bigg(\frac{1}{m}-\frac{1}{m+1}\bigg)+\frac{1}{[x]}\sum_{n\leq x}f(n)-\frac{1}{x}\sum_{n\leq x}f(n)\label{switch_sum}
\end{align}
where we obtain~\eqref{switch_sum} via a simple reordering of the sum. Cancellation (from the telescoping series) now gives that
\begin{align*}
\int_1^x\frac{1}{t^2}\sum_{n\leq t}f(n)\;dt &= \sum_{n\leq [x]-1}\frac{f(n)}{n}+\frac{1}{[x]}f([x])-\frac{1}{x}\sum_{n\leq x}f(n)\\
&= \sum_{n\leq x}\frac{f(n)}{n}-\frac{1}{x}\sum_{n\leq x}f(n),
\end{align*}
which is equivalent to~\eqref{log_identity}.

Now suppose that $\lim_{x\to\infty}\frac{1}{x}\sum_{n\leq x}f(n) = \alpha$. By~\eqref{log_identity},
\begin{align}
\lim_{x\to\infty}\frac{1}{\log{x}}\sum_{n\leq x}\frac{f(n)}{n} &= \lim_{x\to\infty}\bigg(\frac{1}{x\log{x}}\sum_{n\leq x}f(n)+\frac{1}{\log{x}}\int_1^x\frac{1}{t}\sum_{n\leq t}f(n)\frac{dt}{t}\bigg)\notag\\
&= \lim_{x\to\infty}\frac{1}{\log{x}}\int_1^x\frac{1}{t}\sum_{n\leq t}f(n)\frac{dt}{t}.\label{alpha_integrand}
\end{align}
Given any $\epsilon >0$, there is an $x_0:=x_0(\epsilon)$ such that $|\frac{1}{x}\sum_{n\leq x}f(n)-\alpha|<\epsilon$ for $x>x_0$. Applying this inequality to the integrand in~\eqref{alpha_integrand}, for $x>x_0$ we have that
\begin{align}
\int_1^x\frac{1}{t}\sum_{n\leq t}f(n)\frac{dt}{t} &= \int_1^{x_0}\frac{1}{t}\sum_{n\leq t}f(n)\frac{dt}{t}+\int_{x_0}^x\frac{1}{t}\sum_{n\leq t}f(n)\frac{dt}{t}\notag\\
&= M(\epsilon)+\int_{x_0}^x\frac{1}{t}\sum_{n\leq t}f(n)\frac{dt}{t}\notag\\
&= M(\epsilon)+\int_{x_0}^x\bigg(\alpha+\frac{1}{t}\sum_{n\leq t}f(n)-\alpha\bigg)\frac{dt}{t}\notag\\
&= M(\epsilon)+\alpha(\log{x}-\log{x_0})+\int_{x_0}^x\bigg(\frac{1}{t}\sum_{n\leq t}f(n)-\alpha\bigg)\frac{dt}{t}.\label{bound}
\end{align}
From~\eqref{bound} it immediately follows that
\begin{align}
\bigg|\frac{1}{\log{x}}\sum_{n\leq x}\frac{f(n)}{n}-\alpha\bigg| &< \frac{|M(\epsilon)|+|\alpha|\log{x_0}}{\log{x}}+\frac{\epsilon}{\log{x}}\int_{x_0}^x\frac{dt}{t}\notag\\
&= \frac{|M(\epsilon)|+|\alpha|\log{x_0}}{\log{x}}+\epsilon\bigg(1-\frac{\log{x_0}}{\log{x}}\bigg)\label{log_lim}
\end{align}
for $x>x_0$. The desired result now follows from~\eqref{log_lim} since we can make the right-hand side arbitrarily small for all sufficiently large $x$.
\end{proof}

\begin{ex}{8}\label{eight}
Let $f$ be an arithmetic function obeying the crude bound $f(n) = O(n^{o(1)})$, and let $\alpha$ be a complex number. If the logarithmic density of $f$ exists and is equal to $\alpha$, show that $(s-1)\mathcal{D}f(s)\to\alpha$ as $s\to 1^+$, or in other words that $\mathcal{D}f(s)=\frac{\alpha}{s-1}+o\big(\frac{1}{s-1}\big)$ as $s\to 1^+$.
\end{ex}

\begin{proof}
We begin by establishing the following identity:
\begin{align}
\frac{1}{n^s} &= \frac{1}{n}\int_n^\infty\frac{s-1}{x^s}\;dx\label{ns_identity}
\end{align}
for $s>1$.
Beginning with the right-hand side of~\eqref{ns_identity}, we get
\begin{align*}
\frac{1}{n}\int_n^\infty\frac{s-1}{x^s}\;dx &= \lim_{R\to\infty}\frac{s-1}{n}\int_n^R\frac{1}{x^s}\;dx\\
&= \lim_{R\to\infty}\frac{-1}{nx^{s-1}}\Big|_n^R\\
&= \frac{1}{n^s}.
\end{align*}
Now, by~\eqref{ns_identity}
\begin{align}
\mathcal{D}f(s) &= \lim_{N\to\infty} \sum_{n\leq N}\frac{f(n)}{n^s}\notag\\
&= \lim_{N\to\infty} \sum_{n\leq N}\frac{f(n)}{n}\int_n^\infty\frac{s-1}{x^s}\;dx\notag\\
&= \lim_{N\to\infty} \bigg(\int_1^N\frac{s-1}{x^s}\sum_{n\leq x}\frac{f(n)}{n}\;dx+\int_N^\infty\frac{s-1}{x^s}\sum_{n\leq N}\frac{f(n)}{n}\;dx\bigg)\label{int_sum_change}\\
&= \lim_{N\to\infty} \bigg(\int_1^N\frac{s-1}{x^s}\sum_{n\leq x}\frac{f(n)}{n}\;dx+\frac{1}{N^{s-1}}\sum_{n\leq N}\frac{f(n)}{n}\bigg)\label{eval_int}.
\end{align}
We obtain~\eqref{int_sum_change} by rearranging the order of the sum and integral. Note that in~\eqref{int_sum_change}, the sum inside the second integral does not depend on $x$. This gives~\eqref{eval_int}.

For any $s>1$, let $0<\epsilon<s-1$ be given. Then, as in the previous proof, there is an $N_0:=N(\epsilon)$ such that
\begin{align}
\frac{1}{N^{s-1}}\sum_{n\leq N}\frac{f(n)}{n} &= \frac{1}{N^{s-1}}\sum_{n\leq N_0}\frac{f(n)}{n}+O_\epsilon\bigg(\frac{1}{N^{s-1}}\sum_{N_0<n\leq N}\frac{1}{n^{1-\epsilon}}\bigg)\notag\\
&= O_\epsilon\bigg(\frac{1}{N^{s-1}}+N^{\epsilon-(s-1)}\bigg).\label{no_contribution}
\end{align}
Note that~\eqref{no_contribution} goes to 0 as $N\to\infty$ regardless of the choice of all sufficiently small $\epsilon$. In particular, this term will not contribute anything to the limit in~\eqref{eval_int}. This reduces our problem to showing that
\begin{align*}
\lim_{s\to1^+}\int_1^\infty\frac{(s-1)^2}{x^s}\sum_{n\leq x}\frac{f(n)}{n}\;dx &= \alpha.
\end{align*}

Using the same type of argument as in Exercise~\ref{eight}, for any $\epsilon'>0$ there is an $N_1:=N_1(\epsilon')$ such that
\begin{align*}
\int_1^N\frac{s-1}{x^s}\sum_{n\leq x}\frac{f(n)}{n}\;dx &= \int_1^N\frac{(s-1)\log{x}}{x^s}\bigg(\alpha+\frac{1}{\log{x}}\sum_{n\leq x}\frac{f(n)}{n}-\alpha\bigg)\;dx\\ &= \alpha K(s,N)+M(\epsilon')+\int_{N_1}^N\frac{(s-1)\log{x}}{x^s}\bigg(\frac{1}{\log{x}}\sum_{n\leq x}\frac{f(n)}{n}-\alpha\bigg)\;dx.
\end{align*}
where $K(s,N):=\int_{1}^N\frac{(s-1)\log{x}}{x^s}\;dx$. Notice that $K(s,N)$ actually converges as $N\to\infty$ for any $s>1$. With this in mind, we define $K(s):=\lim_{N\to\infty}K(s,N)$. So
\begin{align}
\bigg|\int_1^\infty\frac{(s-1)^2}{x^s}\sum_{n\leq x}\frac{f(n)}{n}\;dx -\alpha\bigg| &< (s-1)\bigg(|M(\epsilon')|+\epsilon ' K(s)+|\alpha|\cdot\bigg|K(s)-\frac{1}{s-1}\bigg|\bigg).\label{final_form_limit}
\end{align}
Therefore, by~\eqref{final_form_limit}, to get the desired result, it is enough to show that $(s-1)K(s,N)$ converges as $N\to\infty$. A simple calculation shows that
\begin{align*}
(s-1)K(s,N) &=\int_1^N\frac{(s-1)^2\log{x}}{x^s}\;dx\\
&= -\frac{(s-1)\log{x}+1}{x^{s-1}}\bigg|_1^N\\
&= 1-\frac{(s-1)\log{N}+1}{N^{s-1}},
\end{align*}
which converges to 1 as $N\to\infty$. This shows that the inequality in~\eqref{final_form_limit} can be made arbitrarily small as $s\to1^+$.
\end{proof}

\begin{ex}{9}\label{nine}
\begin{enumerate}
\item[(i)] For any integer $k\geq 0$, show that
\begin{align*}
\sum_{n\leq x}\frac{\log^k{n}}{n}=Q_k(\log{x})+O_k\bigg(\frac{\log^k{(2+x)}}{x}\bigg)
\end{align*}
for all $x\geq 1$, where $Q_k(t)$ is a polynomial with leading term $\frac{1}{k+1}t^{k+1}$.
\item[(ii)] More generally, for any integers $k,l\geq0$, show that
\begin{align*}
\sum_{n\leq x}\frac{\log^k{n}\log^l{\frac{x}{n}}}{n}=Q_{k,l}(\log{x})+O_{k,l}\bigg(\frac{\log^{k+l}{(2+x)}}{x}\bigg)
\end{align*}
for all $x\geq 1$, where $Q_{k,l}(t)$ is a polynomial with leading term $\frac{k!l!}{(k+l+1)!}t^{k+l+1}$.
\end{enumerate}
\end{ex}

\begin{proof}
As in Exercise~\ref{four}, we prove Exercise~\ref{nine} by induction on $l$. Statement $(i)$ is simply the base case where $l=0$, while $(ii)$ is the general case. We also give an explicit formula for $Q_{k,l}(t)$, namely
\begin{align}
Q_{k,l}(t) &= \frac{k!l!}{(k+l+1)!}t^{k+l+1}+(-1)^k\sum_{i=0}^l\binom{l}{i}(k+i)!t^{l-i}.\label{qkl_poly}
\end{align}
(Note that $Q_k(t)=Q_{k,0}(t)$.)

Using~\eqref{log_identity} and~\eqref{four_result}, we have that
\begin{align}
\sum_{n\leq x}\frac{\log^k{n}}{n} &= \frac{1}{x}\sum_{n\leq x}\log^k{n}+\int_1^x\frac{1}{t}\sum_{n\leq t}\log^k{n}\frac{dt}{t}\notag\\
&= P_{k,0}(\log{x})+\int_1^xP_{k,0}(\log{t})\frac{dt}{t}+O_k\bigg(\frac{\log^k(2+x)}{x}\bigg).\label{log_poly_int}
\end{align}
Applying~\eqref{pkl_poly} (with $l=0$) and integration by substitution on~\eqref{log_poly_int},
\begin{align*}
\int_1^xP_{k,0}(\log{t})\frac{dt}{t} &= \int_1^x\sum_{i=0}^k(-1)^i\frac{k!}{(k-i)!}\log^{k-i}{t}\frac{dt}{t}\\
&= \sum_{i=0}^k(-1)^i\frac{k!}{(k+1-i)!}\log^{k+1-i}{x}.
\end{align*}
So
\begin{align*}
P_{k,0}(\log{x})+\int_1^xP_{k,0}(\log{t})\frac{dt}{t} &= \sum_{i=0}^k(-1)^i\frac{k!}{(k-i)!}\log^{k-i}{x}\bigg(\frac{\log{x}}{k+1-i}+1\bigg)\\
&=\frac{1}{k+1}\log^{k+1}{x}+(-1)^kk!\\
&= Q_k(\log{x})
\end{align*}
as desired in this case.

For the general case where we have $(ii)$ for some $l-1\geq 0$, it follows by assumption that
\begin{align*}
\sum_{n\leq x}\frac{\log^k{n}\log^l{\frac{x}{n}}}{n} &= \log{x}\sum_{n\leq x}\frac{\log^k{n}\log^{l-1}{\frac{x}{n}}}{n} -\sum_{n\leq x}\frac{\log^{k+1}{n}\log^{l-1}{\frac{x}{n}}}{n}\\
&= \log{x}\cdot Q_{k,l-1}(\log{x})-Q_{k+1,l-1}(\log{x})+O_{k,l}\bigg(\frac{\log^{k+l}(2+x)}{x}\bigg).
\end{align*}
In other words, to complete the proof, it is enough to show that
\begin{align*}
Q_{k,l}(t) &= t\cdot Q_{k,l-1}(t)-Q_{k+1,l-1}(t).
\end{align*}
From~\eqref{qkl_poly}, we have
\begin{align*}
t\cdot Q_{k,l-1}(t)-Q_{k+1,l-1}(t) &= \frac{k!(l-1)!}{(k+l)!}t^{k+l+1}+(-1)^k\sum_{i=0}^{l-1}\binom{l-1}{i}(k+i)!t^{l-i}\\
& \qquad\qquad -\frac{(k+1)!(l-1)!}{(k+l+1)!}t^{k+l+1}+(-1)^k\sum_{i=0}^{l-1}\binom{l-1}{i}(k+1+i)!t^{l-1-i}\\
&= \frac{k!(l-1)!}{(k+l)!}\bigg(1-\frac{k+1}{k+l+1}\bigg)t^{k+l+1}+(-1)^kk!t^l\\
& \qquad\qquad +(-1)^k\sum_{i=1}^{l-1}\bigg(\binom{l-1}{i}+\binom{l-1}{i-1}\bigg)(k+i)!t^{l-i}\\
&= Q_{k,l}(t),
\end{align*}
which gives $(ii)$.
\end{proof}

\begin{ex}{10}\label{ten}
Show rigorously that for any non-negative integer $k\geq 0$ and real $s>1$, the Riemann zeta function $\zeta$ is $k$-times differentiable at $s$ and one has
\begin{align*}
\zeta^{(k)}(s) &= (-1)^k\sum_{n=1}^\infty\frac{\log^k{n}}{n^s}\\
&= (-1)^k\frac{k!}{(s-1)^{k+1}}+O_k(1).
\end{align*}
\end{ex}

\begin{proof}
\phantom{a}
\end{proof}

\begin{ex}{11}\label{eleven}
Let $y<x$ be real numbers, and let $f:[y,z]\to\mathbb{C}$ be a continuously differentiable function. Show that
\begin{align*}
\sum_{n\in\mathbb{Z}:y\leq n\leq x}f(n)=\int_y^xf(t)\;dt+O\bigg(\int_y^x|f'(t)|\;dt+|f(y)|\bigg).
\end{align*}
Conclude that if $t$ is a non-zero number, that the function $n\mapsto n^{\imath t}$ has logarithmic density zero, but does not have a natural density.
\end{ex}

\begin{proof}
Let $\{t\}=t-[t]$. The sum on the left-hand side of Exercise~\ref{eleven} is equivalent to a Stieltjes integral with respect to the floor function $[t]$.
\begin{align}
\sum_{y\leq n\leq x}f(n) &= \int_y^xf(t)\;d[t]\notag\\
&= \int_y^xf(t)\;dt-\int_y^xf(t)\;d\{t\}\label{two_int}.
\end{align}
To finish the proof, it is enough to show that the integral with respect to $d\{t\}$ is bounded appropriately. Using integration by parts,
\begin{align*}
\int_y^xf(t)\;d\{t\} &= f(x)\{x\}-f(y)\{y\}-\int_y^xf'(t)\{t\}\;dt\\
&= O\bigg(\int_y^x|f'(t)|\;dt+|f(x)|+|f(y)|\bigg)
\end{align*}
as desired.

For the function $f_t(n)=n^{\imath t}$, where $t$ is a non-zero number, we have
\begin{align*}
\frac{1}{\log{x}}\sum_{n\leq x}\frac{f_t(n)}{n} &= \frac{1}{\log{x}}\sum_{n\leq x}\frac{1}{n^{1-\imath t}}\\
&= \frac{1}{\log{x}}\int_1^x\frac{1}{y^{1-\imath t}}\;dy+O\bigg(\frac{|1-\imath t|}{\log{x}}\int_1^x\frac{1}{y^2}\;dy+\frac{1}{x\log{x}}+\frac{1}{\log{x}}\bigg)\\
&= O_t\bigg(\frac{1}{\log{x}}\bigg)
\end{align*}
by the result of Exercise~\ref{eleven}. In particular, this shows that the logarithmic density of $f_t(n)$ is zero.

For any $\epsilon>0$, we have
\begin{align}
\sum_{x\leq n\leq (1+\epsilon)x}f_t(n) &= \sum_{x\leq n\leq (1+\epsilon)x}n^{\imath t}\notag\\
&= \int_x^{(1+\epsilon)x}y^{\imath t}\;dy+O\bigg(\int_x^{(1+\epsilon)x}\frac{1}{y}\;dy+1\bigg)\notag\\
&= \frac{((1+\epsilon)^{1+\imath t}-1)x^{1+\imath t}}{1+\imath t}+O_\epsilon(1)\label{cauchy_sum}
\end{align}
by Exercise~\ref{eleven}. If $f_t(n)$ has a natural density, then the limit of $\frac{1}{x}\sum_{x\leq n\leq (1+\epsilon)x}f_t(n)$ must be zero as $x\to\infty$. However,~\eqref{cauchy_sum} shows that this cannot be the case. It follows that $f_t(n)$ does not have a natural density.
\end{proof}

\end{document}
